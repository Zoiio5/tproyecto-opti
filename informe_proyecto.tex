\documentclass[12pt]{article}
\usepackage[utf8]{inputenc}
\usepackage[spanish]{babel}
\usepackage{amsmath, amssymb}
\usepackage{graphicx}
\usepackage{booktabs}
\usepackage{hyperref}
\usepackage{enumitem} 
\usepackage{geometry}
\usepackage{algorithm}
\usepackage{amsmath}
\usepackage{algpseudocode}
\usepackage{listings}
\geometry{a4paper, margin=2.5cm}

\title{Proyecto Optimización}
\author{Grupo 5}
\date{\today}

\begin{document}

\maketitle


\section{Definiciones del Modelo}

\subsection*{Conjuntos}
\begin{itemize}
\item $N$: Conjunto de todos los nodos.
\item $P \subset N$: Plantas de tratamiento.
\item $T \subset N$: Tanques.
\item $C_1 \subset N$: Clientes de transbordo.
\item $C_2 \subset N$: Clientes finales.
\item $A \subseteq N\times N$: Conjunto de arcos posibles.
\item $D$: Conjunto de diámetros disponibles.
\end{itemize}

\subsection*{Parametros}
\begin{itemize}
\item $u_d$: Capacidad máxima de flujo para el diámetro $d \in D$.
\item $c_{ad}$: Costo de instalación para arco $a \in A$ y diámetro $d$.
\item $t_a$: Costo de transporte por unidad de flujo en el arco $a \in A$.
\item $\text{demanda}_i$: Demanda del nodo $i \in C_1 \cup C_2$ (0 para otros nodos).
\end{itemize}

\subsection*{Variables}
\begin{itemize}

\item $y_{ad} = 
\begin{cases} 
1, & \text{si se instala tubería de diámetro } d \text{ en el arco } a\\
0, & \text{si no}
\end{cases}$

\item $f_{a}$: Flujo de agua por el arco $a$.
\end{itemize}


\section{Función Objetivo}
\begin{equation*}
\min z =  \sum_{a \in A} \sum_{d \in D} c_{ad} \cdot y_{ad} + \sum_{a \in A} t_a \cdot f_a
\end{equation*}


\section{Restricciones}
\begin{enumerate}

\item \textbf{Asignación unica de diámetro por arco:}
\begin{equation*}
\sum_{d \in D} y_{ad} \leq 1 \quad \forall a \in A
\end{equation*}

\item \textbf{Capacidad de flujo segun diametro instalado:}
\begin{equation*}
f_a \leq \sum_{d \in D} u_d \cdot y_{ad} \quad \forall a \in A
\end{equation*}

\item \textbf{Conservación de flujo en tanques (sin demanda):}
\[
    \sum_{(j,i) \in A} f_{ji} = \sum_{(i,j) \in A} f_{ij}
\] 

\item \textbf{Balance en clientes de transbordo:}
 \[
    \sum_{(j,i) \in A} f_{ji} = \text{demanda}_i + \sum_{(i,j) \in A} f_{ij}
\] 


\item \textbf{Balance en clientes finales:}
 \[
    \sum_{(j,i) \in A} f_{ji} = \text{demanda}_i
\] 

\item \textbf{Flujo no negativo:}
\begin{equation*}
f_a \geq 0 \quad \forall a \in A
\end{equation*}\\


\item \textbf{Las plantas deben generar el flujo suficiente para satisfacer la demanda total:}
\[
\sum_{i \in P} \sum_{(i,j) \in A} f_{ij} \geq \sum_{k \in C_1 \cup C_2} \text{demanda}_k
\]

\end{enumerate}

\vspace{7cm}


\section{Descripción de Función Objetivo y Restricciones}

\begin{itemize}
    \item \textbf{Función Objetivo:} Minimiza el costo total del sistema, considerando la suma de los costos de instalación de tuberías y los costos de transporte por el flujo de agua.\\
    
    \item \textbf{Restricción 1:} Asegura que en cada arco se instale a lo más una tubería (de un solo diámetro).\\

    \item \textbf{Restricción 2:} Garantiza que el flujo por un arco no supere la capacidad del diámetro instalado en ese arco. \\


    \item \textbf{Restricción 3:} En los tanques, el flujo que entra debe ser igual al que sale, ya que no tienen demanda. \\

    \item \textbf{Restricción 4:} En los clientes de transbordo, el flujo entrante debe cubrir su demanda propia y el flujo que deben reenviar a los clientes finales. \\

    \item \textbf{Restricción 5:} En los clientes finales, el flujo que reciben debe ser igual exactamente a su demanda (no redistribuyen). \\

    \item \textbf{Restricción 6:} El flujo en cualquier arco debe ser no negativo. \\

    \item \textbf{Restricción 7:} Las plantas deben inyectar al sistema un flujo total al menos igual a la demanda total de todos los clientes (de transbordo y finales).
\end{itemize}
\vspace{8cm}


\section{Descripción del proceso de generación de las instancias}

El proceso de generación de instancias para el problema de optimización de redes de distribución se ha diseñado con el objetivo principal de garantizar la \textbf{satisfacibilidad} de las soluciones generadas. A continuación se detalla la metodología implementada.

\subsection{Parámetros de Configuración}

\subsubsection{Especificaciones de Diámetros}
Las tuberías disponibles se clasifican en tres categorías con sus respectivas capacidades máximas de flujo:

\begin{table}[h]
\centering
\begin{tabular}{@{}lcc@{}}
\toprule
\textbf{Diámetro} & \textbf{Diámetro Nominal (mm)} & \textbf{Flujo Máximo} \\
\midrule
D2 & 75 & 795 \\
D3 & 100 & 1,414 \\
D5 & 150 & 3,181 \\
\bottomrule
\end{tabular}
\caption{Especificaciones técnicas de los diámetros disponibles}
\end{table}

\subsubsection{Costos de Instalación}
Se consideran dos tipos de terreno con costos diferenciados:

\begin{table}[h]
\centering
\begin{tabular}{@{}lccc@{}}
\toprule
\textbf{Tipo de Terreno} & \textbf{D2} & \textbf{D3} & \textbf{D5} \\
\midrule
Tipo A (favorable) & 20 & 24 & 32 \\
Tipo B (difícil) & 50 & 62 & 78 \\
\bottomrule
\end{tabular}
\caption{Costos de instalación por tipo de terreno}
\end{table}

\subsubsection{Rangos por Tamaño de Instancia}
Las instancias se clasifican en tres categorías según su complejidad:

\begin{table}[h]
\centering
\begin{tabular}{@{}lcccc@{}}
\toprule
\textbf{Tamaño} & \textbf{Plantas} & \textbf{Tanques} & \textbf{Transbordo} & \textbf{Finales} \\
\midrule
Pequeñas & 2--3 & 5--8 & 5--8 & 8--15 \\
Medianas & 3--5 & 8--15 & 8--15 & 15--30 \\
Grandes & 4--6 & 15--30 & 20--35 & 30--60 \\
\bottomrule
\end{tabular}
\caption{Rangos de nodos por categoría de instancia}
\end{table}

\subsection{Algoritmo de Generación}

\begin{algorithm}[H]
\caption{Generación de Instancias Satisfacibles}
\begin{enumerate}
    \item \textbf{Requerimientos:}
    \begin{itemize}
        \item Tamaño de instancia $\tau \in \{\text{pequeña, mediana, grande}\}$
        \item Índice de instancia $idx$
    \end{itemize}
    
    \item \textbf{Resultado esperado:} Instancia satisfacible $\mathcal{I}$
    
    \item Generar número de nodos: $n_P, n_T, n_{C1}, n_{C2}$ según rangos de $\tau$
    
    \item Generar demandas balanceadas: $d_1, d_2$ con factor $f_\tau$
    
    \item Construir topología conectada: $A_{PT}, A_{TC1}, A_{C1C2}$
    
    \item Calcular suministro con holgura: 
    \[
    s = \frac{(\sum d_1 + \sum d_2) \cdot 1.3}{n_P}
    \]
    
    \item Generar costos de transporte: $c_{trans} \sim \mathcal{N}(6, 1.5^2)$
    
    \item Asignar costos de instalación alternando tipos A/B
    
    \item Validar capacidades de flujo vs. demanda total
    
    \item \textbf{return} Instancia $\mathcal{I}$ con metadatos de validación
\end{enumerate}
\end{algorithm}
\subsection{Generación de Demandas}

Las demandas se generan siguiendo una estrategia \textbf{conservadora} para garantizar satisfacibilidad:

\begin{align}
d_{base} &\sim U(30, 80) \\
d_i &= \text{máx}(20, d_{base} \cdot f_\tau \cdot U(0.8, 1.2))
\end{align}

donde $f_\tau$ es el factor de moderación por tamaño:
\begin{itemize}
    \item $f_{\text{pequeña}} = 0.6$
    \item $f_{\text{mediana}} = 0.7$ 
    \item $f_{\text{grande}} = 0.8$
\end{itemize}

\subsection{Construcción de Topología Conectada}

La topología se construye garantizando conectividad completa entre todos los niveles:

\subsubsection{Conexiones Plantas-Tanques}
\begin{itemize}[leftmargin=*]
    \item Cada tanque se conecta a 1--2 plantas mínimo
    \item Se verifica que ninguna planta quede aislada
    \item Redundancia para balancear cargas
\end{itemize}

\subsubsection{Conexiones Tanques-Centros de Transbordo}
\begin{itemize}[leftmargin=*]
    \item Cada tanque se conecta a 1--3 centros de transbordo
    \item Distribución uniforme de conexiones
\end{itemize}

\subsubsection{Conexiones Transbordo-Destinos Finales}
\begin{itemize}[leftmargin=*]
    \item \textbf{Garantía}: Cada destino final tiene al menos una conexión
    \item Conexiones adicionales para redundancia (1--2 por centro)
\end{itemize}

\subsection{Cálculo de Suministro con Holgura}

El suministro se calcula para garantizar factibilidad con un margen de seguridad:

\begin{equation}
S_{total} = \left(\sum_{i=1}^{n_{C1}} d_{1,i} + \sum_{j=1}^{n_{C2}} d_{2,j}\right) \cdot \lambda
\end{equation}

donde $\lambda = 1.3$ es el factor de holgura y:

\begin{equation}
s_p = \frac{S_{total}}{n_P} \quad \forall p \in P
\end{equation}

\subsection{Validación de Capacidades}

Se implementa un mecanismo de validación para verificar que las capacidades de flujo sean adecuadas:

\begin{enumerate}
    \item \textbf{Requerimientos:}
    \begin{itemize}
        \item Conjunto de arcos $A$
        \item Demanda total $D_{total}$
    \end{itemize}
    
    \item \textbf{Resultado esperado:} Validación de factibilidad
    
    \item Calcular número total de arcos:
    \[
    |A| = |A_{PT}| + |A_{TC1}| + |A_{C1C2}|
    \]
    
    \item Estimar flujo promedio:
    \[
    \bar{f} = \frac{D_{total}}{|A| \cdot 0.7}
    \]
    
    \item Si $\bar{f} > \text{capacidad}_{D3}$ entonces:
    \begin{itemize}
        \item (Acción de validación a tomar en caso de que la condición se cumpla)
    \end{itemize}
    
    \item Verificar distribución de diámetros disponibles
\end{enumerate}


\subsection{Generación de Costos}

\subsubsection{Costos de Transporte}
Los costos de transporte siguen una distribución normal truncada:

\begin{equation}
c_{trans,i} = \text{máx}(1.0, X_i) \quad \text{donde } X_i \sim \mathcal{N}(6, 1.5^2)
\end{equation}

\subsubsection{Costos de Instalación}
Se alternan los tipos de terreno para cada arco:

\begin{equation}
\text{tipo}(i) = \begin{cases}
\text{A} & \text{si } i \bmod 2 = 0 \\
\text{B} & \text{si } i \bmod 2 = 1
\end{cases}
\end{equation}

\subsection{Métricas de Satisfacibilidad}

Para cada instancia generada se calculan las siguientes métricas de validación:

\begin{itemize}[leftmargin=*]
    \item \textbf{Factor de holgura}: $\frac{S_{total}}{D_{total}} \geq 1.3$
    \item \textbf{Conectividad}: Todos los destinos alcanzables desde plantas
    \item \textbf{Balance de capacidades}: Flujo estimado vs. capacidades máximas
    \item \textbf{Distribución de costos}: Rango y variabilidad de costos
\end{itemize}

\subsection{Formato de Salida}

Las instancias se generan en formato MiniZinc (.dzn) con la siguiente estructura:

\begin{lstlisting}[language=Pascal, basicstyle=\small\ttfamily, frame=single]
% Parametros de tamano
nP = <num_plantas>;
nT = <num_tanques>;  
nC1 = <num_transbordo>;
nC2 = <num_finales>;
nA = <num_arcos>;
nD = 3;

% Topologia
arc_from = [<indices_origen>];
arc_to = [<indices_destino>];

% Ofertas y demandas  
supply = [<suministros_por_nodo>];
demand = [<demandas_por_nodo>];

% Capacidades y costos
max_capacity = [795, 1414, 3181];
install_cost = array2d(1..nA, 1..nD, [<matriz_costos>]);
trans_cost = [<costos_transporte>];
\end{lstlisting}

\subsection{Resultados Esperados}

Con esta metodología de generación se espera obtener:

\begin{itemize}[leftmargin=*]
    \item \textbf{Alta tasa de satisfacibilidad}: $> 95\%$ de instancias factibles
    \item \textbf{Balance oferta-demanda}: Factor de holgura consistente $\geq 1.3$
    \item \textbf{Conectividad garantizada}: Todos los nodos destino alcanzables
    \item \textbf{Diversidad de soluciones}: Variabilidad en topologías y costos
    \item \textbf{Escalabilidad controlada}: Complejidad progresiva por tamaño
\end{itemize}

La implementación de este proceso asegura que las instancias generadas sean tanto \textbf{realistas} como \textbf{computacionalmente tratables}, proporcionando un conjunto de pruebas robusto para algoritmos de optimización de redes de distribución.

\vspace{16cm}



\section{Análisis de Factibilidad del Modelo}

La factibilidad del modelo depende de varios parámetros que determinan si existe al menos una solución que satisface todas las restricciones. A continuación se clasifican los parámetros según su impacto en la factibilidad:

\subsection*{Parámetros que afectan directamente la factibilidad}

Estos parámetros son críticos para la existencia de soluciones viables:

\begin{itemize}
    \item \textbf{Demanda ($\text{demanda}_i$):} Si la suma de las demandas de los nodos en $C_1 \cup C_2$ supera la capacidad total disponible en la red, el problema será inviable.
    
    \item \textbf{Capacidad de los diámetros ($u_d$):} Si los diámetros disponibles no permiten transportar suficiente caudal a través de los arcos, no será posible abastecer la demanda.

    \item \textbf{Arcos disponibles ($A \subseteq N \times N$):} La conectividad de la red es fundamental. Si no existen caminos desde las plantas hacia los nodos de consumo, el modelo será inviable.

    \item \textbf{Plantas ($P$):} Son los únicos nodos que generan flujo. Si hay muy pocas plantas o no están conectadas adecuadamente, no se podrá satisfacer la demanda total.

    \item \textbf{Diámetros disponibles ($D$):} Si los tipos de tuberías disponibles no tienen la capacidad necesaria para cubrir los requerimientos de flujo, no habrá solución factible.
\end{itemize}

\subsection*{Parámetros que no afectan la factibilidad}

Estos parámetros influyen únicamente en el valor del costo total del sistema, pero no impiden que se encuentre una solución:

\begin{itemize}
    \item \textbf{Costo de instalación ($c_{ad}$):} Afecta la función objetivo, pero no limita la capacidad de transportar flujo.
    
    \item \textbf{Costo de transporte ($t_a$):} Incide en el costo total, pero no interviene en las restricciones de factibilidad.

    \item \textbf{Tipo de costo utilizado (a o b):} Modifica el costo asociado a cada arco y diámetro, pero no altera la posibilidad de cumplir con la demanda.
\end{itemize}

% --- PRESENTACIÓN DE RESULTADOS Y GRÁFICOS ---
Los resultados obtenidos por la aplicación se almacenan en la carpeta \texttt{reportes/}, organizados por tamaño de instancia (\texttt{grandes}, \texttt{medianas}, \texttt{pequeñas}). Cada archivo de reporte contiene la solución encontrada para una instancia específica, así como información relevante sobre el proceso de resolución (valor de la función objetivo, tiempo de cómputo, factibilidad, etc.).

A continuación, se resumen los principales resultados obtenidos:

\begin{itemize}
    \item \textbf{Instancias pequeñas:} Se resolvieron 10 instancias con un promedio de 2.5 plantas, 6 tanques, 6 clientes de transbordo y 11 clientes finales. La demanda total promedio es de 150 m³/día.
    \item \textbf{Instancias medianas:} Se resolvieron 10 instancias con un promedio de 4 plantas, 10 tanques, 10 clientes de transbordo y 20 clientes finales. La demanda total promedio es de 300 m³/día.
    \item \textbf{Instancias grandes:} Se resolvieron 10 instancias con un promedio de 5 plantas, 20 tanques, 30 clientes de transbordo y 45 clientes finales. La demanda total promedio es de 600 m³/día.
\end{itemize}

Se logró una alta tasa de satisfacibilidad, con más del 95\% de las instancias siendo factibles bajo los parámetros y restricciones del modelo.


\end{document}
